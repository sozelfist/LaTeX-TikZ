% source Tcolorbox manual
% modified by Truong Nhan Nguyen

\documentclass[tikz, border=10pt]{standalone}

\usepackage[most]{tcolorbox}
\usepackage{amsmath, amssymb, amsfonts, mathtools}

\tcbset{
    frogbox/.style={
        enhanced,
        colback=green!10, colframe=green!65!black,
        enlarge top by=5.5mm,
        overlay={
            \foreach \x in {2cm, 3.5cm}{
                \begin{scope}[shift={([xshift=\x]frame.north west)}]
                    \path[draw=green!65!black, fill=green!10, line width=1mm] (0,0) arc (0:180:5mm);
                    \path[fill=black] (-0.2, 0) arc (0:180:1mm);
                \end{scope}
            }
        }
    }
}

\begin{document}
    \begin{tcolorbox}[frogbox, title=A Frogbox, fonttitle=\sffamily\bfseries]
        \sffamily
        To calculate the horizontal position the kinematic differential equations are needed:\\
        \begin{align}
            \dot{u}=u\cos{\psi}-v\sin{\psi}\\
            \dot{e}=u\sin{\psi}+v\cos{\psi}
        \end{align}
        For small angles the following approximation can be used:\\
        \begin{align}
            \dot{n}=u-v\delta_{\psi}\\
            \dot{e}=u\delta_{\psi}+v
        \end{align}
    \end{tcolorbox}
\end{document}